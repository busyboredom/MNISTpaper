\documentclass[transmag]{IEEEtran}
\usepackage{latexsym}
\usepackage{graphicx}
\usepackage{amsfonts,amssymb,amsmath}
\usepackage{hyperref}
\usepackage{fancyvrb}
\usepackage{float}
\def\BibTeX{{\rm B\kern-.05em{\sc i\kern-.025em b}\kern-.08em T\kern-.1667em\lower.7ex\hbox{E}\kern-.125emX}}

\begin{document}

\title{Classification of Handwritten Digits using Convolutional Neural Networks
in Tensorflow 2.0}

\author{Charlie Wilkin
\thanks{Submitted December 1st, 2019. This work was completed as part of a 
final project submission for EE456, Intro to Neural Networks.}}

\IEEEtitleabstractindextext{\begin{abstract}
Objective:
Computer vision is a rapidly developing field with a significant
dependence on hardware acceleration and parallel processing. While massively
parallel hardware has existed in some capacity for decades, powerful libraries
enabling its use by untrained indeviduals are a relatively new phenomen. The
launch of Tensorflow 1.0 by Google in 2017 (and more recently, Tensorflow 2.0)
pushed AI-oriented hardware acceleration closer to the mainstream. In this 
project, the beginner-friendly interface of Tensorflow is demonstrated by 
achieving greater than 99\% accuracy on the popular MNIST dataset using fewer
than 50 lines of Python code on consumer hardware.
\end{abstract}

\begin{IEEEkeywords}
Neural Networks, Computer Vision, Parallel Processing, Convolutional Neural Networks
\end{IEEEkeywords}}

\maketitle

\section{Introduction}

\IEEEPARstart{T}{his} project demonstrates one method of achieving greater than
99\% accuracy on the MNIST dataset of handwritten digits. While this
performance is far from state of the art, the implementation is easy to
understand and trains quickly on mid-range consumer hardware.  
dure outlined in Section II will assume some rudamentary experience with the
Python programming language, a working install of Tensorflow 2.0 on
a Debian-based system such as Ubuntu, and a compatible Nvidia Graphics
Processing Unit. 

If you do not currently have Tensorflow 2.0 installed with GPU support, you
can follow the directions provided by Google at 
\href{https://www.tensorflow.org/install/gpu}{www.tensorflow.org/install/pip}.
The same webpage outlines the installation of necessary dependencies including
Nvidia drivers, the CUDA toolkit, and the cuDNN software development kit. 
Implementation details regarding the operation of layers within the proposed
model and preparation of the dataset are explained thorougly in Section II.

\section{Operation and Implementation}



\subsection{Preparing the Dataset}

The MNIST (Modified National Institute of Standards and Technology) dataset
is a collection of 70,000 handwritten digits prepared in 1999 by researchers at
Courant Institute and Google Labs \cite{ref1}. Each image in the MNIST dataset
contains 28x28 grayscale pixels representing a number from 0-9. The images are
centered on the digit and size-normalized, with one-hot encoded labels. Fig. 1
shows one image from the MNIST dataset.


\begin{figure}[H]
  \centerline{\includegraphics[width=1.5in]{fig1}}
  \caption{A handwritten digit drom the MNIST dataset, consisting of 28x28
  grayscale pixels\label{fig1}}
\end{figure}

The MNIST dataset is included in Tensorflow, and can be loaded easily by first
importing Tenslorflow and then storing the dataset in a variable.

\begin{figure}[H]
\begin{Verbatim}[samepage=true]
from __future__ import absolute_import, \
                       division, \
                       print_function, \
                       unicode_literals \

import tensorflow as tf

# Store the MNIST dataset
mnist = tf.keras.datasets.mnist
\end{Verbatim}
\end{figure}

The dataset is then divided into seperate training and testing sets.
This allows us to verify the model's performance by testing it on data that it
was not exposed to during training. To accelerate the training process, the 
grayscale pixel values are also normalized to between 0 and 1 by 
dividing each pixel by 255.

\begin{figure}[H]
\begin{Verbatim}
(x_train, y_train), (x_test, y_test) = \
                         mnist.load_data()

x_train, x_test = x_train / 255.0,
                  x_test / 255.0
\end{Verbatim}
\end{figure}

\noindent Finally, a color channel dimension is added to the images. This is
dimension is blank, and exists to tell Tensorflow that there are no color
channels in these images.

\begin{figure}[H]
\begin{Verbatim}
x_train = x_train[..., tf.newaxis]
x_test = x_test[..., tf.newaxis]
\end{Verbatim}
\end{figure}

\subsection{Defining the Model}

The model is defined sequantially, layer by layer. The first layer consists
a collection of 32 feature detectors, each 3x3 elements, convolved (i.e. 
scanned across) the image \cite{ref2}. The resulting output is a set of 32 
matrixes, each containing 26x26 elements. To compress this output into 
something more manageable, a 2x2 pooling layer is applied to keep only one out 
of every four pixels before feeding the result into a second layer of 64 
convolutions \cite{ref3}.

\begin{figure}[H]
\begin{Verbatim}
model = tf.keras.models.Sequential([

    tf.keras.layers.Conv2D(
            32, \
            (3, 3), \
            activation='relu', \
            input_shape=(28, 28, 1)),

    tf.keras.layers.MaxPooling2D((2, 2)),

    tf.keras.layers.Conv2D(
            64, \
            (3, 3), \
            activation='relu'),
\end{Verbatim}
\end{figure}

The activation function applied at each element of the 3x3 feature detectors
is known as the ReLU function, or Rectified Linear Unit \cite{ref4}. The ReLU activation 
function (shown in Fig. 2) is popular for its simplicity and its 
easily-calculated derivative. The use of a non-linear function is crucial, as
any sum of purely linear functions will necessarily collapse into a single 
linear function itself.

\begin{figure}[H]
  \includegraphics[width=3in]{fig2}
  \caption{The Rectified Linear Unit activation function used throughout 
  this model.\label{fig2}}
\end{figure}

The output of the second convolutional layer is a three dimensional tensor of
shape 11x11x64, and the output of the network as a whole must be a ten 
dimensional output vector. To achieve this, the output of the second 
convolutional layer is first flattened into a single 7744 dimensional vector.
This vector is then fed into a traditional, fully-connected layer of 64 units
with the ReLU activation function.

\begin{figure}[H]
\begin{Verbatim}
    tf.keras.layers.Flatten(),
    tf.keras.layers.Dense(
            64, \
            activation='relu'),
\end{Verbatim}
\end{figure}

Up to this point, a total of 514,496 trainable parameters have been added to
the model. Bearing in mind that the MNIST dataset contains only 70,000 images,
a model of this size risks fitting to random, unintended patterns in the 
training data and failing to generalize to the test data as a result. To
prevent this, the network is forced to act as a consensus of smaller networks
by randomly eliminating 20\% of the connections in the fully-connected layer
during training.

\begin{figure}[H]
\begin{Verbatim}
    tf.keras.layers.Dropout(0.2),
\end{Verbatim}
\end{figure}


The final layer consists of a ten fully-connected units with
the softmax activation function, which compresses the outputs into a values
between 0 and 1 while constraining them such that they add to 1. This causes
the output vector to resemble a probability distribution, where the largest
value represents the predicted digit. 

\begin{figure}[H]
\begin{Verbatim}
    tf.keras.layers.Dense(
            10, \
            activation='softmax')
])
\end{Verbatim}
\end{figure}

\noindent A summary of the complete model, generated using Tensorflow's
\verb model.summary() method, appears in Table I. 

\begin{table}[H]
  \caption{Summary of Model, as provided by Tensorflow\label{table}}
  \centering
  \begin{BVerbatim}
___________________________________________________
Layer (type)       Output Shape            Param #
===================================================
conv2d          (None, 26, 26, 32)        320
___________________________________________________
max_pooling2d   (None, 13, 13, 32)        0
___________________________________________________
conv2d_1        (None, 11, 11, 64)        18496
___________________________________________________
flatten         (None, 7744)              0
___________________________________________________
dense           (None, 64)                495680
___________________________________________________
dropout         (None, 64)                0
___________________________________________________
dense_1         (None, 10)                650
===================================================
Total params: 515,146
Trainable params: 515,146
Non-trainable params: 0
___________________________________________________
  \end{BVerbatim}
\end{table}

\subsection{Training the Model}
Format and save your graphic images using a suitable graphics processing 
program that will allow you to create the images as PostScript (PS), 
Encapsulated PostScript (EPS), or Tagged Image File Format (TIFF), sizes 
them, and adjusts the resolution settings. If you created your source files 
in one of the following you will be able to submit the graphics without 
converting to a PS, EPS, or TIFF file: Microsoft Word, Microsoft PowerPoint, 
Microsoft Excel, or Portable Document Format (PDF). 

\subsubsection*{Sizing of Graphics}

Most charts graphs and tables are one column wide (3 1/2 inches or 21 picas) 
or two-column width (7 1/16 inches, 43 picas wide). We recommend that you 
avoid sizing figures less than one column wide, as extreme enlargements may 
distort your images and result in poor reproduction. Therefore, it is better 
if the image is slightly larger, as a minor reduction in size should not 
have an adverse affect the quality of the image. 

\subsubsection*{Size of Author Photographs}

The final printed size of an author photograph is exactly 
1 inch wide by 1 1/4 inches long (6 picas~$\times$~7 1/2 picas). Please 
ensure that the author photographs you submit are proportioned similarly. If 
the author's photograph does not appear at the end of the paper, then please 
size it so that it is proportional to the standard size of 1 9/16 inches 
wide by 
2 inches long (9 1/2 picas~$\times$~12 picas). JPEG files are only 
accepted for author photos.

\subsubsection*{Print Color Graphics Requirements}

IEEE accepts color graphics in the following formats: EPS, PS, TIFF, Word, 
PowerPoint, Excel, and PDF. The resolution of a RGB color TIFF file should 
be 400 dpi. 

When sending color graphics, please supply a high quality hard copy or PDF 
proof of each image. If we cannot achieve a satisfactory color match using 
the electronic version of your files, we will have your hard copy scanned. 
Any of the files types you provide will be converted to RGB color EPS files. 

\subsubsection*{Web Color Graphics}

IEEE accepts color graphics in the following formats: EPS, PS, TIFF, Word, 
PowerPoint, Excel, and PDF. The resolution of a RGB color TIFF file should 
be at least 400 dpi. 

Your color graphic will be converted to grayscale if no separate grayscale 
file is provided. If a graphic is to appear in print as black and white, it 
should be saved and submitted as a black and white file. If a graphic is to 
appear in print or on IEEE Xplore in color, it should be submitted as RGB 
color. 

\subsubsection*{Graphics Checker Tool}

The IEEE Graphics Checker Tool enables users to check graphic files. The 
tool will check journal article graphic files against a set of rules for 
compliance with IEEE requirements. These requirements are designed to ensure 
sufficient image quality so they will look acceptable in print. After 
receiving a graphic or a set of graphics, the tool will check the files 
against a set of rules. A report will then be e-mailed listing each graphic 
and whether it met or failed to meet the requirements. If the file fails, a 
description of why and instructions on how to correct the problem will be 
sent. The IEEE Graphics Checker Tool is available at \href 
{http://graphicsqc.ieee.org/}{http://graphicsqc.ieee.org/}

For more Information, contact the IEEE Graphics H-E-L-P Desk by e-mail at 
\href {mailto:graphics@ieee.org}{mailto:graphics@ieee.org}. You will then receive an e-mail response and 
sometimes a request for a sample graphic for us to check.

\begin{figure}
\centerline{\includegraphics[width=3.5in]{fig1}}
\caption{Magnetization as a function of applied field.
It is good practice to explain the significance of the figure in the
caption.\label{fig1}}
\end{figure}

\begin{table}
\caption{Units for Magnetic Properties}
\label{table}
\setlength{\tabcolsep}{3pt}
\begin{tabular}{|p{25pt}|p{75pt}|p{115pt}|}
\hline
Symbol& 
Quantity& 
Conversion from Gaussian and \par CGS EMU to SI $^{\mathrm{a}}$ \\
\hline
$\Phi $& 
magnetic flux& 
1 Mx $\to 10^{-8}$ Wb $= 10^{-8}$ V$\cdot $s \\
$B$& 
magnetic flux density, \par magnetic induction& 
1 G $\to 10^{-4}$ T $= 10^{-4}$ Wb/m$^{2}$ \\
$H$& 
magnetic field strength& 
1 Oe $\to 10^{3}/(4\pi )$ A/m \\
$m$& 
magnetic moment& 
1 erg/G $=$ 1 emu \par $\to 10^{-3}$ A$\cdot $m$^{2} = 10^{-3}$ J/T \\
$M$& 
magnetization& 
1 erg/(G$\cdot $cm$^{3}) =$ 1 emu/cm$^{3}$ \par $\to 10^{3}$ A/m \\
4$\pi M$& 
magnetization& 
1 G $\to 10^{3}/(4\pi )$ A/m \\
$\sigma $& 
specific magnetization& 
1 erg/(G$\cdot $g) $=$ 1 emu/g $\to $ 1 A$\cdot $m$^{2}$/kg \\
$j$& 
magnetic dipole \par moment& 
1 erg/G $=$ 1 emu \par $\to 4\pi \times 10^{-10}$ Wb$\cdot $m \\
$J$& 
magnetic polarization& 
1 erg/(G$\cdot $cm$^{3}) =$ 1 emu/cm$^{3}$ \par $\to 4\pi \times 10^{-4}$ T \\
$\chi , \kappa $& 
susceptibility& 
1 $\to 4\pi $ \\
$\chi_{\rho} $& 
mass susceptibility& 
1 cm$^{3}$/g $\to 4\pi \times 10^{-3}$ m$^{3}$/kg \\
$\mu $& 
permeability& 
1 $\to 4\pi \times 10^{-7}$ H/m \par $= 4\pi \times 10^{-7}$ Wb/(A$\cdot $m) \\
$\mu_{r}$& 
relative permeability& 
$\mu \to \mu_{r}$ \\
$w, W$& 
energy density& 
1 erg/cm$^{3} \to 10^{-1}$ J/m$^{3}$ \\
$N, D$& 
demagnetizing factor& 
1 $\to 1/(4\pi )$ \\
\hline
\multicolumn{3}{p{251pt}}{Vertical lines are optional in tables. Statements that serve as captions for 
the entire table do not need footnote letters.} \\
\multicolumn{3}{p{251pt}}{$^{\mathrm{a}}$Gaussian units are the same as cg emu for magnetostatics; Mx 
$=$ maxwell, G $=$ gauss, Oe $=$ oersted; Wb $=$ weber, V $=$ volt, s $=$ 
second, T $=$ tesla, m $=$ meter, A $=$ ampere, J $=$ joule, kg $=$ 
kilogram, H $=$ henry.}
\end{tabular}
\label{tab1}
\end{table}

\subsection{Copyright Form}
An IEEE copyright form should accompany your final submission. You can get a 
.pdf, .html, or .doc version at \href 
{http://www.ieee.org/copyright}{http://www.ieee.org/copyright}. Authors are responsible for obtaining any 
security clearances.

\section{Units}
Use either SI (MKS) or CGS as primary units. (SI units are strongly 
encouraged.) English units may be used as secondary units (in parentheses). 
\textbf{This applies to papers in data storage.} For example, write ``15 
Gb/cm$^{2}$ (100 Gb/in$^{2})$.'' An exception is when English units are used 
as identifiers in trade, such as ``3$\frac12$-in disk drive.'' Avoid 
combining SI and CGS units, such as current in amperes and magnetic field in 
oersteds. This often leads to confusion because equations do not balance 
dimensionally. If you must use mixed units, clearly state the units for each 
quantity in an equation.

The SI unit for magnetic field strength $H$ is A/m. However, if you wish to use 
units of T, either refer to magnetic flux density $B$ or magnetic field 
strength symbolized as $\mu_{0}H$. Use the center dot to separate 
compound units, e.g., ``A$\cdot$m$^{2}$.''

\section{Helpful Hints}
\subsection{Figures and Tables}
Because IEEE will do the final formatting of your paper, you do not need to 
position figures and tables at the top and bottom of each column. Large 
figures and tables may span both columns. Place figure captions below the 
figures; place table titles above the tables. If your figure has two parts, 
include the labels ``(a)'' and ``(b)'' as part of the artwork. Please verify 
that the figures and tables you mention in the text actually exist. 
\textbf{Please do not include captions as part of the figures. Do not put 
captions in ``text boxes'' linked to the figures. Do not put borders around 
the outside of your figures.} Use the abbreviation ``Fig.'' even at the 
beginning of a sentence. Do not abbreviate ``Table.'' Tables are numbered 
with Roman numerals. 

Figure axis labels are often a source of confusion. Use words rather than 
symbols. As an example, write the quantity ``Magnetization,'' or 
``Magnetization $M$,'' not just ``$M$.'' Put units in parentheses. Do not label 
axes only with units. As in Fig. 1, for example, write ``Magnetization 
(A/m)'' or ``Magnetization (A$\cdot $m$^{-1})$,'' not just ``A/m.'' Do not 
label axes with a ratio of quantities and units. For example, write 
``Temperature (K),'' not ``Temperature/K.'' 

Multipliers can be especially confusing. Write ``Magnetization (kA/m)'' or 
``Magnetization (10$^{3}$ A/m).'' Do not write ``Magnetization (A/m) 
$\times$ 1000'' because the reader would not know whether the top axis 
label in Fig. 1 meant 16000 A/m or 0.016 A/m. Figure labels should be 
legible, approximately 8 to 12 point type.

\subsection{\LaTeX-Specific Advice}

Please use ``soft'' (e.g., \verb|\eqref{Eq}|) cross references instead
of ``hard'' references (e.g., \verb|(1)|). That will make it possible
to combine sections, add equations, or change the order of figures or
citations without having to go through the file line by line.

Please don't use the \verb|{eqnarray}| equation environment. Use
\verb|{align}| or \verb|{IEEEeqnarray}| instead. The \verb|{eqnarray}|
environment leaves unsightly spaces around relation symbols.

Please note that the \verb|{subequations}| environment in {\LaTeX}
will increment the main equation counter even when there are no
equation numbers displayed. If you forget that, you might write an
article in which the equation numbers skip from (17) to (20), causing
the copy editors to wonder if you've discovered a new method of
counting.

{\BibTeX} does not work by magic. It doesn't get the bibliographic
data from thin air but from .bib files. If you use {\BibTeX} to produce a
bibliography you must send the .bib files. 

{\LaTeX} can't read your mind. If you assign the same label to a
subsubsection and a table, you might find that Table I has been cross
referenced as Table IV-B3. 

{\LaTeX} does not have precognitive abilities. If you put a
\verb|\label| command before the command that updates the counter it's
supposed to be using, the label will pick up the last counter to be
cross referenced instead. In particular, a \verb|\label| command
should not go before the caption of a figure or a table.

Do not use \verb|\nonumber| or \verb|\notag| inside the \verb|{array}| environment. It
will not stop equation numbers inside \verb|{array}| (there won't be
any anyway) and it might stop a wanted equation number in the
surrounding equation.

\subsection{References}
Number citations consecutively in square brackets \cite{ref1}. The sentence 
punctuation follows the brackets \cite{ref2}. Multiple references \cite{ref2}, \cite{ref3} are each 
numbered with separate brackets \cite{ref1}--\cite{ref3}. When citing a section in a book, 
please give the relevant page numbers \cite{ref2}. In sentences, refer simply to the 
reference number, as in \cite{ref3}. Do not use ``Ref. \cite{ref3}'' or ``reference \cite{ref3}'' 
except at the beginning of a sentence: ``Reference \cite{ref3} shows $\ldots$ .'' Please 
do not use automatic endnotes in \emph{Word}, rather, type the reference list at the 
end of the paper using the ``References'' style.

Number footnotes separately in superscripts (Insert $\vert$ 
Footnote).\footnote{It is recommended that footnotes be avoided (except for 
the unnumbered footnote with the receipt date on the first page). Instead, 
try to integrate the footnote information into the text.} Place the actual 
footnote at the bottom of the column in which it is cited; do not put 
footnotes in the reference list (endnotes). Use letters for table footnotes 
(see Table I). 

Please note that the references at the end of this document are in the 
preferred referencing style. Give all authors' names; do not use ``\emph{et al}.'' 
unless there are six authors or more. Use a space after authors' initials. 
Papers that have not been published should be cited as ``unpublished'' \cite{ref4}. 
Papers that have been accepted for publication, but not yet specified for an 
issue should be cited as ``to be published'' \cite{ref5}. Papers that have been 
submitted for publication should be cited as ``submitted for publication'' 
\cite{ref6}. Please give affiliations and addresses for private communications \cite{ref7}.

Capitalize only the first word in a paper title, except for proper nouns and 
element symbols. For papers published in translation journals, please give 
the English citation first, followed by the original foreign-language 
citation \cite{ref8}.

\subsection{Abbreviations and Acronyms}
Define abbreviations and acronyms the first time they are used in the text, 
even after they have already been defined in the abstract. Abbreviations 
such as IEEE, SI, ac, and dc do not have to be defined. Abbreviations that 
incorporate periods should not have spaces: write ``C.N.R.S.,'' not ``C. N. 
R. S.'' Do not use abbreviations in the title unless they are unavoidable 
(for example, ``IEEE'' in the title of this article).

\subsection{Equations}
Number equations consecutively with equation numbers in parentheses flush 
with the right margin, as in (\ref{eq1}). First use the equation editor to create 
the equation. Then select the ``Equation'' markup style. Press the tab key 
and write the equation number in parentheses. To make your equations more 
compact, you may use the solidus ( / ), the exp function, or appropriate 
exponents. Use parentheses to avoid ambiguities in denominators. Punctuate 
equations when they are part of a sentence, as in
\begin{multline}
\int_{0}^{r_{2}} {F(r,\phi )} \,dr\,d\phi =[\sigma r_{2} /(2\mu_{0} )] \\ 
 \cdot \int_{0}^{\infty} \exp (-\lambda \vert 
z_{j} -z_{i} \vert )\lambda^{-1}J_{1} (\lambda r_{2} )J_{0} 
(\lambda r_{i} )\,d\lambda . 
\label{eq1}
\end{multline}
Be sure that the symbols in your equation have been defined before the 
equation appears or immediately following. Italicize symbols ($T$ might refer 
to temperature, but T is the unit tesla). Refer to ``(\ref{eq1}),'' not ``Eq. (\ref{eq1})'' 
or ``equation (\ref{eq1}),'' except at the beginning of a sentence: ``Equation (\ref{eq1}) 
is $\ldots$ .''

\subsection{Other Recommendations}
Use one space after periods and colons. Hyphenate complex modifiers: 
``zero-field-cooled magnetization.'' Avoid dangling participles, such as, 
``Using (\ref{eq1}), the potential was calculated.'' [It is not clear who or what 
used (\ref{eq1}).] Write instead, ``The potential was calculated by using (\ref{eq1}),'' or 
``Using (\ref{eq1}), we calculated the potential.''

Use a zero before decimal points: ``0.25,'' not ``.25.'' Use ``cm$^{3}$,'' 
not ``cc.'' Indicate sample dimensions as ``0.1 cm~$\times$~0.2 cm,'' not 
``0.1~$\times$~0.2 cm$^{2}$.'' The abbreviation for ``seconds'' is ``s,'' 
not ``sec.'' Do not mix complete spellings and abbreviations of units: use 
``Wb/m$^{2}$'' or ``webers per square meter,'' not ``webers/m$^{2}$.'' When 
expressing a range of values, write ``7 to 9'' or ``7-9,'' not 
``7$\sim$9.''

A parenthetical statement at the end of a sentence is punctuated outside of 
the closing parenthesis (like this). (A parenthetical sentence is punctuated 
within the parentheses.) In American English, periods and commas are within 
quotation marks, like ``this period.'' Other punctuation is ``outside''! 
Avoid contractions; for example, write ``do not'' instead of ``don't.'' The 
serial comma is preferred: ``A, B, and C'' instead of ``A, B and C.''

If you wish, you may write in the first person singular or plural and use 
the active voice (``I observed that $\ldots$'' or ``We observed that $\ldots$'' 
instead of ``It was observed that $\ldots$''). Remember to check spelling. If 
your native language is not English, please get a native English-speaking 
colleague to carefully proofread your paper.

\section{Some Common Mistakes}
The word ``data'' is plural, not singular. The subscript for the 
permeability of vacuum $\mu_{0}$ is zero, not a lowercase letter ``o.'' 
The term for residual magnetization is ``remanence''; the adjective is 
``remanent''; do not write ``remnance'' or ``remnant.'' Use the word 
``micrometer'' instead of ``micron.'' A graph within a graph is an 
``inset,'' not an ``insert.'' The word ``alternatively'' is preferred to the 
word ``alternately'' (unless you really mean something that alternates). Use 
the word ``whereas'' instead of ``while'' (unless you are referring to 
simultaneous events). Do not use the word ``essentially'' to mean 
``approximately'' or ``effectively.'' Do not use the word ``issue'' as a 
euphemism for ``problem.'' When compositions are not specified, separate 
chemical symbols by en-dashes; for example, ``NiMn'' indicates the 
intermetallic compound Ni$_{0.5}$Mn$_{0.5}$ whereas ``Ni--Mn'' indicates an 
alloy of some composition Ni$_{x}$Mn$_{1-x}$.

Be aware of the different meanings of the homophones ``affect'' (usually a 
verb) and ``effect'' (usually a noun), ``complement'' and ``compliment,'' 
``discreet'' and ``discrete,'' ``principal'' (e.g., ``principal 
investigator'') and ``principle'' (e.g., ``principle of measurement''). Do 
not confuse ``imply'' and ``infer.'' 

Prefixes such as ``non,'' ``sub,'' ``micro,'' ``multi,'' and ``ultra'' are 
not independent words; they should be joined to the words they modify, 
usually without a hyphen. There is no period after the ``et'' in the Latin 
abbreviation ``\emph{et al.}'' (it is also italicized). The abbreviation ``i.e.,'' means 
``that is,'' and the abbreviation ``e.g.,'' means ``for example'' (these 
abbreviations are not italicized).

An excellent style manual and source of information for science writers is 
\cite{ref9}. A general IEEE style guide and an \emph{Information for Authors} are both available at \href 
{http://www.ieee.org/web/publications/authors/transjnl/index.html}{http://www.ieee.org/web/publications/authors/transjnl/index.html}

\section{Editorial Policy}
Submission of a manuscript is not required for participation in a 
conference. Do not submit a reworked version of a paper you have submitted 
or published elsewhere. Do not publish ``preliminary'' data or results. The 
submitting author is responsible for obtaining agreement of all coauthors 
and any consent required from sponsors before submitting a paper. IEEE 
TRANSACTIONS and JOURNALS strongly discourage courtesy authorship. It is the 
obligation of the authors to cite relevant prior work.

The Transactions and Journals Department does not publish conference records 
or proceedings. The TRANSACTIONS does publish papers related to conferences 
that have been recommended for publication on the basis of peer review. As a 
matter of convenience and service to the technical community, these topical 
papers are collected and published in one issue of the TRANSACTIONS.

At least two reviews are required for every paper submitted. For 
conference-related papers, the decision to accept or reject a paper is made 
by the conference editors and publications committee; the recommendations of 
the referees are advisory only. Undecipherable English is a valid reason for 
rejection. Authors of rejected papers may revise and resubmit them to the 
TRANSACTIONS as regular papers, whereupon they will be reviewed by two new 
referees.

\section{Publication Principles}
The contents of IEEE TRANSACTIONS and JOURNALS are peer-reviewed and 
archival. The TRANSACTIONS publishes scholarly articles of archival value as 
well as tutorial expositions and critical reviews of classical subjects and 
topics of current interest. 

Authors should consider the following points:

\begin{enumerate}
\item Technical papers submitted for publication must advance the state of knowledge and must cite relevant prior work. 
\item The length of a submitted paper should be commensurate with the importance, or appropriate to the complexity, of the work. For example, an obvious extension of previously published work might not be appropriate for publication or might be adequately treated in just a few pages.
\item Authors must convince both peer reviewers and the editors of the scientific and technical merit of a paper; the standards of proof are higher when extraordinary or unexpected results are reported. 
\item Because replication is required for scientific progress, papers submitted for publication must provide sufficient information to allow readers to perform similar experiments or calculations and use the reported results. Although not everything need be disclosed, a paper must contain new, useable, and fully described information. For example, a specimen's chemical composition need not be reported if the main purpose of a paper is to introduce a new measurement technique. Authors should expect to be challenged by reviewers if the results are not supported by adequate data and critical details.
\item Papers that describe ongoing work or announce the latest technical achievement, which are suitable for presentation at a professional conference, may not be appropriate for publication in a TRANSACTIONS or JOURNAL.
\end{enumerate}

\section{Conclusion}
Please include a brief summary of the possible clinical implications of your 
work in the conclusion section. Although a conclusion may review the main 
points of the paper, do not replicate the abstract as the conclusion. 
Consider elaborating on the translational importance of the work or suggest 
applications and extensions. 

\section*{Appendix}

Appendixes, if needed, appear before the acknowledgment.

\section*{Acknowledgment}

The preferred spelling of the word ``acknowledgment'' in American English is 
without an ``e'' after the ``g.'' Use the singular heading even if you have 
many acknowledgments. Avoid expressions such as ``One of us (S.B.A.) would 
like to thank $\ldots$ .'' Instead, write ``F. A. Author thanks $\ldots$ .'' 
\textbf{Sponsor and financial support acknowledgments are placed in the 
unnumbered footnote on the first page, not here.}

\begin{thebibliography}{00}

\bibitem{ref1} Y. LeCun, C. Cortes, C. J. Burges, MNIST Handwritten Digits 
    Database; AT\&T Bell Laboratories, 2010.

\bibitem{ref2} Y. LeCun, B. Boser, J. S. Denker, D. Henderson, R. E. Howard, 
    W. Hubbard, L. D. Jackel, Backpropagation Applied to Handwritten Zip Code 
    Recognition; AT\&T Bell Laboratories, 1989.

\bibitem{ref3} Yamaguchi, Kouichi; Sakamoto, Kenji; Akabane, Toshio; Fujimoto, 
    Yoshiji (November 1990). A Neural Network for Speaker-Independent Isolated 
    Word Recognition. First International Conference on Spoken Language 
    Processing (ICSLP 90). Kobe, Japan.

\bibitem{ref4} Hahnloser, R.; Sarpeshkar, R.; Mahowald, M. A.; Douglas, 
    R. J.; Seung, H. S. (2000). "Digital selection and analogue amplification 
    coexist in a cortex-inspired silicon circuit". Nature. 405: 947–951.


\bibitem{ref5} E. H. Miller, ``A note on reflector arrays (Periodical style---Accepted for publication),'' \emph{IEEE Trans. Antennas Propagat.}, to be published.
\bibitem{ref6} J. Wang, ``Fundamentals of erbium-doped fiber amplifiers arrays (Periodical style---Submitted for publication),'' \emph{IEEE J. Quantum Electron.}, submitted for publication.
\bibitem{ref7} C. J. Kaufman, Rocky Mountain Research Lab., Boulder, CO, private communication, May 1995.
\bibitem{ref8} Y. Yorozu, M. Hirano, K. Oka, and Y. Tagawa, ``Electron spectroscopy studies on magneto-optical media and plastic substrate interfaces (Translation Journals style),'' \emph{IEEE Transl. J. Magn.Jpn.}, vol. 2, Aug. 1987, pp. 740--741 [\emph{Dig. 9}$^{th}$\emph{ Annu. Conf. Magnetics} Japan, 1982, p. 301].
\bibitem{ref9} M. Young, \emph{The Technical Writers Handbook.} Mill Valley, CA: University Science, 1989.
\bibitem{ref10} J. U. Duncombe, ``Infrared navigation---Part I: An assessment of feasibility (Periodical style),'' \emph{IEEE Trans. Electron Devices}, vol. ED-11, pp. 34--39, Jan. 1959.
\bibitem{ref11} S. Chen, B. Mulgrew, and P. M. Grant, ``A clustering technique for digital communications channel equalization using radial basis function networks,'' \emph{IEEE Trans. Neural Networks}, vol. 4, pp. 570--578, Jul. 1993.
\bibitem{ref12} R. W. Lucky, ``Automatic equalization for digital communication,'' \emph{Bell Syst. Tech. J.}, vol. 44, no. 4, pp. 547--588, Apr. 1965.
\bibitem{ref13} S. P. Bingulac, ``On the compatibility of adaptive controllers (Published Conference Proceedings style),'' in \emph{Proc. 4th Annu. Allerton Conf. Circuits and Systems Theory}, New York, 1994, pp. 8--16.
\bibitem{ref14} G. R. Faulhaber, ``Design of service systems with priority reservation,'' in \emph{Conf. Rec. 1995 IEEE Int. Conf. Communications,} pp. 3--8.
\bibitem{ref15} W. D. Doyle, ``Magnetization reversal in films with biaxial anisotropy,'' in \emph{1987 Proc. INTERMAG Conf.}, pp. 2.2-1--2.2-6.
\bibitem{ref16} G. W. Juette and L. E. Zeffanella, ``Radio noise currents n short sections on bundle conductors (Presented Conference Paper style),'' presented at the IEEE Summer power Meeting, Dallas, TX, Jun. 22--27, 1990, Paper 90 SM 690-0 PWRS.
\bibitem{ref17} J. G. Kreifeldt, ``An analysis of surface-detected EMG as an amplitude-modulated noise,'' presented at the 1989 Int. Conf. Medicine and Biological Engineering, Chicago, IL.
\bibitem{ref18} J. Williams, ``Narrow-band analyzer (Thesis or Dissertation style),'' Ph.D. dissertation, Dept. Elect. Eng., Harvard Univ., Cambridge, MA, 1993. 
\bibitem{ref19} N. Kawasaki, ``Parametric study of thermal and chemical nonequilibrium nozzle flow,'' M.S. thesis, Dept. Electron. Eng., Osaka Univ., Osaka, Japan, 1993.
\bibitem{ref20} J. P. Wilkinson, ``Nonlinear resonant circuit devices (Patent style),'' U.S. Patent 3 624 12, July 16, 1990. 
\bibitem{ref21} \emph{IEEE Criteria for Class IE Electric Systems} (Standards style)$,$ IEEE Standard 308, 1969.
\bibitem{ref22} \emph{Letter Symbols for Quantities}, ANSI Standard Y10.5-1968.
\bibitem{ref23} R. E. Haskell and C. T. Case, ``Transient signal propagation in lossless isotropic plasmas (Report style),'' USAF Cambridge Res. Lab., Cambridge, MA Rep. ARCRL-66-234 (II), 1994, vol. 2.
\bibitem{ref24} E. E. Reber, R. L. Michell, and C. J. Carter, ``Oxygen absorption in the Earth's atmosphere,'' Aerospace Corp., Los Angeles, CA, Tech. Rep. TR-0200 (420-46)-3, Nov. 1988.
\bibitem{ref25} (Handbook style) \emph{Transmission Systems for Communications,} 3rd ed., Western Electric Co., Winston-Salem, NC, 1985, pp. 44--60.
\bibitem{ref26} \emph{Motorola Semiconductor Data Manual,} Motorola Semiconductor Products Inc., Phoenix, AZ, 1989.
\bibitem{ref27} (Basic Book/Monograph Online Sources) J. K. Author. (year, month, day). \emph{Title} (edition) [Type of medium]. Volume (issue). Available: \underline {http://www.(URL})
\bibitem{ref28} J. Jones. (1991, May 10). Networks (2nd ed.) [Online]. Available: \underline {http://www.atm.com}
\bibitem{ref29} (Journal Online Sources style) K. Author. (year, month). Title. \emph{Journal} [Type of medium]. Volume(issue), paging if given. Available: \underline {http://www.(URL})
\bibitem{ref30} R. J. Vidmar. (1992, August). On the use of atmospheric plasmas as electromagnetic reflectors. \emph{IEEE Trans. Plasma Sci.} [Online]. \emph{21(3).} pp. 876--880. Available: http://www.halcyon.com/pub/journals/21ps03-vidmar
\end{thebibliography}

\end{document}
